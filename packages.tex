

%%%% PACKAGES AUTOMATTICALLY PROVIDED BY CLASS 
%%%%
%%%%  (No need to reload.  Options can be provided to these packages using the
%%%%   \PassOptionsToPackage command as per example at end.)

%% ifthen     -   Provides simple boolean commands

%% ifpdf      -   Detects whether pdflatex is being used.

%% graphicx   -   Allows inclusion of graphics in eps or jpg/pdf format

%% geometry   -   A more modern way of setting the page margins.
%%                'report' class options are passed automatically.

%% setspace   -   Define line spacing

%% appendix   -   Required for appendices

%% ae         -   Nicer pdf output using T1 fonts

%% fontenc    -   T1 encoding stops some errors for unknown fonts

%% titlesec   -   Custom chapter titles and page headers

%% caption    -   Custom formatting of captions

%% subcaption -   Use subfigures in captions and formatting of such.

%% tocloft    -   Allow modifications to the table of contents and lists

%% mathptmx   -   Option 'msfonts' to use these alternative font packages.
%% helvet         Note: you could add your own font packages in this file.

%% showlabels -   Print labels on the page if 'labels' option is given.

%% hyperef    -   Put clickable links into the document

%% hypcap     -   Links to figures show the figures rather than just the
%%                caption.

%% nomencl    -   Loaded with option 'nomencl' this provides a ``fly-by-wire''
%%                nomenclature indexing system. See chapter3.tex for an
%%                example.

%% xcolor     -   Loaded with option 'dvipsnames' with provides colors for pdf
%%                output. The options auto adjust for pdflatex and hyperref.
%%                See http://en.wikibooks.org/wiki/LaTeX/Colors for colours.

%%%%%%%%%%%%%%%%%%%%%%%%%%%%%%%%%%%%%%%%%%%%%%%%%%%%%%%%%%%%%%%%%%%%%%%%%%
%%%%%          Put ADDITIONAL packages you want to use here           %%%%
%%%%%%%%%%%%%%%%%%%%%%%%%%%%%%%%%%%%%%%%%%%%%%%%%%%%%%%%%%%%%%%%%%%%%%%%%%

%% Typset URLs properly. This package does automatic breaking of long URLs.
%% If you're using hyperref you dont need this but you might need the
%% 'breakurl' package depending on your compilation route.
%% Note: 'breakurl' must be the last package (which is normally 'hyperref').
\usepackage[hyphens]{url} 

%% AMS Math Packages
\usepackage{amsmath}
\usepackage{amssymb}
\usepackage{amsthm}
\usepackage{mathtools}


%% Better fractions ie 1/2 with diagonal slash
\usepackage{nicefrac}

%% Nicer tables
\usepackage{booktabs}
\usepackage{tabularx,makecell,rotating,lscape}
\usepackage{longtable}
\usepackage{xltabular}
\usepackage{dcolumn}
\usepackage{multirow}
\usepackage{adjustbox}

%% Citations -- follows the Harvard style
\usepackage[backend=bibtex,maxcitenames=2,style=authoryear-icomp,sortcites=false]{biblatex}
\usepackage{csquotes}

%% Proper hyphenation
\usepackage[british]{babel}

%% Rotating
\usepackage{pdflscape}

%% Paragraph control
\usepackage[defaultlines=4,all]{nowidow}

%% for pseudocode
\usepackage[ruled,vlined]{algorithm2e}

%% Additional packages
\usepackage{enumitem}
\usepackage{siunitx}
\usepackage{float}
\usepackage{color}
\usepackage{changebar}
\usepackage{acronym} 

%% TO create random text - delete where appropriate
\usepackage{blindtext}